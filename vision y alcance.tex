\documentclass{article}

% Language setting
% Replace `english' with e.g. `spanish' to change the document language
\usepackage[spanish]{babel}

% Set page size and margins
% Replace `letterpaper' with`a4paper' for UK/EU standard size
\usepackage[letterpaper,top=2cm,bottom=2cm,left=3cm,right=3cm,marginparwidth=1.75cm]{geometry}

% Useful packages
\usepackage{amsmath}
\usepackage{graphicx}
\usepackage[colorlinks=true, allcolors=blue]{hyperref}

\title{Documento de vision y alcance "ley de Ohm"}
\author{Ulises Cortes San Miguel}
\begin{document}
\maketitle

\begin{abstract}
Ingenieria de software
\end{abstract}

\begin{center}
\section{Generalidades}
\end{center}

\subsection{Introducción}

En el presente documento se identifican los diferentes aspectos
correspondientes a la visión y alcance del proyecto Testing Calculator.

La aplicación web Testing Calculator se plantea como una solución para obtener el calculo de la "ley de Ohm" de manera rapida.

\subsection{Alcance}
En el presente documento se pretende mostrar las limitaciones, condiciones y
requerimientos que se necesitan exponer y analizar para el correcto desarrollo
del proyecto Testing Calculator, de igual manera se desea mostrar las
características para el desarrollo de software

\begin{center}
\section{Caracteristicas del producto}
\end{center}

\subsection{Caracteristicas principales}

En el desarrollo del proyecto se desempeñarán las mejores prácticas usando
técnicas y herramientas, permitiendo un óptimo desempeño al trabajar en el desarrollo del aplicativo, tales como:

\begin{itemize}
    \item Aplicación desarrollada en ambiente web.
    \item Uso del framework de desarrollo Angular.
    \item Bootstrap
    \item Pruebas de testing al producto atraves de Jasmine y Karma.
    \item Uso de Docker para el despliegue en Google Cloud
    \item Uso de la ingeniería de software para la toma de requerimientos.
    \item Validación final (producto final contra requerimientos y diseños expuestos en el presente documento).
\end{itemize}

\end{document}